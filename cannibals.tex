% A good introduction to latex can be found here:
%    http://www.cse.ohio-state.edu/~hank/latex/lshort141.pdf

\documentclass[a4paper]{report}


\usepackage{listings}  %  needed for source code listings
\lstset{language=Java}         

% set the document title, author, and date here.
%  once set, the \maketitle command (within the document)
%  will display them nicely
\title{Missionaries and Cannibals Solution}
\author{Delos Chang}

\begin{document}
\maketitle

\section{Introduction}
%States are either legal, or not legal. First, give an upper bound on the number of states, without considering legality of states. (Hint -- 331 is one state. 231 another, although illegal. Keep counting.) Describe how you got this number.
%Use a drawing program such as inkscape to draw part of the graph of states, including at least the first state, all actions from that state, and all actions from those first-reached states. Show which of these states are legal and which aren't (for example, by using the color of the nodes). Include and describe this figure in your report.

\section{Implementation of the model}
%Present the work in your report. Include the key parts of your code using the listings environment in LaTeX, and discuss how the code works. Also describe your testing and convince the reader that your testing process demonstrated correctness of your code.

The model is implemented in 
\verb`CannibalProblem.java`.  Here's my code for \verb`getSuccessors`:

% Small trick -- dealing with tabs.  
%  When I copied from eclipse, there were large tabs in the pasted
%  text that made the following listing not fit on the page.  I did 
%  a search-and-replace, by copying a tab from this document into 
%  the "search" entry in the dialog box, and replacing with two spaces.

\begin{lstlisting}
public ArrayList<UUSearchNode> getSuccessors() {
  // the final array to be returned
  ArrayList<UUSearchNode> retArr = new ArrayList<UUSearchNode>();
  
  int boatPlace = state[2];
  int candMissionaries = -1;
  int candCannibals = -1;
  int candBoat = -1;
  
  // Determine which missionaries and cannibals can travel
  if (boatPlace == 1){
    System.out.println("Boat @ starting side");
    // the candidate missionaries that can travel on the boat
    candMissionaries = state[0];
    candCannibals = state[1];
    candBoat = 0;
  } else if (boatPlace == 0){
    candMissionaries = totalMissionaries - state[0];
    candCannibals = totalCannibals - state[1];
    candBoat = 1;
  } else {
    // not valid input for boat
    System.out.println("Boat place not valid: " + boatPlace);
    System.exit(1);
  }
  
  for(int missCount=candMissionaries; missCount>=0; missCount--){
    for(int cannCount=candCannibals; cannCount>=0; cannCount--){
      System.out.println("Checking ("+missCount+","+cannCount+")");
      CannibalNode possNode;
      
      // Must fit in boat and have at least one missionary rowing
      if (missCount + cannCount <= BOAT_SIZE && (missCount > 0 || cannCount > 0)){
        // Must have something happen
        if (missCount == 0 && cannCount == 0){
          continue;
        }
        
        if (boatPlace == 1){
          // boat on starting side so starting side is subtracted
          possNode = new CannibalNode(candMissionaries - missCount, 
              candCannibals - cannCount, candBoat, depth+1);
        } else {
          // boat on other side so starting side is added
          possNode = new CannibalNode( state[0] + missCount, 
              state[1] + cannCount, candBoat, depth+1);
        }
        System.out.println("("+possNode.state[0]+","+possNode.state[1]+","+possNode.state[2]+")");
      } else {
        continue;
      }
      
      boolean isSafe = isSafeState(possNode);
      if (isSafe){
        // State is valid, add to the array
        System.out.println("Adding " + possNode);
        retArr.add(possNode);
      } else {
        // Node was not a valid state
        System.out.println("Not a safe state!");
      }
    }
  }
  return retArr;
}
\end{lstlisting}

The basic idea of \verb`getSuccessors` is that it returns an array of valid states
based off of the node that was passed in. So, given the first start node: (331), it will
run through each combination of missionaries and cannibals. So it checks 3 Missionaries and 
3 Cannibals on the boat, then 3 Missionaries 2 Cannibals, then 3 Missionaries 1 Cannibal, 
3 Missionaries 0 Cannibals, 2 Missionaries 1 Cannibal and so on. 

Then, it will check if 
that combination is valid for the boat size. If it is, it will add or subtract (depending
on where the boat is) to create the new state. This new state is passed into the \verb`isSafeState`
to verify whether the missionaries are safe. 

I used a method \verb`isSafeState` that returns \verb`true` if the missionaries do not get
eaten by the cannibals. We can check this by first validating that there are missionaries on
either side of the river. Then we MUST make sure that the cannibals cannot outnumber
missionaries on either side. 

If there are no missionaries on one side, that implies that all the missionaries are on the 
other side. Thus, we just need to check whether the missionaries are outnumbered on the other side. 
This will cover the edge case where there are no missionaries but more than 0 cannibals on the same side. 
Without this edge case, the algorithm would not correctly process a node like (031).

\begin{lstlisting}
// checks whether the humans get eaten :(
private boolean isSafeState(CannibalNode node){
  // miss + cannibals on starting side
  int startMissionaries = node.state[0];
  int startCannibals = node.state[1];
  // miss + cannibals on other side
  int otherMissionaries = totalMissionaries - startMissionaries;
  int otherCannibals = totalCannibals - startCannibals;
  
  if (startMissionaries != 0 && otherMissionaries != 0){
    // must have more missionaries than cannibals or else eaten :(
    if (startMissionaries >= startCannibals && 
        otherMissionaries >= otherCannibals){
      return true;
    }
  }
  
  // If no missionaries are on one side, must mean missionaries are on other side
  // Therefore, we check if the missionaries are outnumbered on the other side
  // Also starting with 1 Cannibal and 0 Missionaries on starting side is NOT 
  // a valid state
  if (otherMissionaries == 0){
    return startMissionaries >= startCannibals;
  }
  
  if (startMissionaries == 0){
    return otherMissionaries >= otherCannibals;
  }
  
  // not a safe state
  return false;
}

\end{lstlisting}

\section{Breadth-first search}
%Test, present, and discuss your code in the report. A good test would at the very least check the breadth-first search with respect to the nodes you drew in the intro.
%Show the solution output by your code for missionaries and cannibals.


\section{Memoizing depth-first search}

\section{Path-checking depth-first search}

\section{Iterative deepening search}



\section{Lossy missionaries and cannibals}



\end{document}
